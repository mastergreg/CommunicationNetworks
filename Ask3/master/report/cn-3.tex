%{{{ preamble
\documentclass[a4paper,9pt]{article}
\usepackage{anysize}
\marginsize{2cm}{2cm}{1cm}{1cm}
%\textwidth 6.0in \textheight = 664pt
\usepackage{xltxtra}
\usepackage{xunicode}
\usepackage{graphicx}
\usepackage{color}
\usepackage{xgreek}
\usepackage{fancyvrb}
\usepackage{minted}
\usepackage{listings}
\usepackage{enumitem} 
\usepackage{framed} 
\usepackage{relsize}
\usepackage{float} 
\usepackage{pstricks}
\usepackage{pst-node}
\usepackage{pst-blur}
\setmainfont[Mapping=tex-text]{FreeSerif}
%}}}
\begin{document}

\def\thesection {\Roman{section}.}

\include{title/title}

\section
\begin{figure}[h]
	\centering
	\includegraphics[width=0.5\textwidth]{files/1.png}
\end{figure}

	
	
Τρέχουμε το animation και παρατηρούμε την εξέλιξη της πορείας μετάδοσης των πακέτων:
\begin{itermize}
\item Αρχικά, για t<0.5sec, δεν έχουμε ροή πακέτων.
\item Για 0.5sec<t< 1 sec, έχουμε ροή πακέτων μόνο από τον 5 στον 7, μέσω
της μικρότερης διαδρομής (5\rightarrow 1 \rightarrow 2 \rightarrow 7) με κόκκινα πακέτα.
\item Για 1sec<t<2sec έχουμε και ροή πράσινων πακέτων από τον κόμβο 7 προς τον
5, ακολουθώντας την αντίστροφη διαδρομή από προηγουμένως.
\item Η μετάδοση των κόκκινων πακέτων διακόπτεται σε χρόνο 2sec, ενώ η
μετάδοση των πράσινων πακέτων σε 2.5sec. Τέλος, για t=3sec, ολοκληρώνεται η
όλη εφαρμογή, μέσα από τη διαδικασία finish.
\end{itermize}

\begin{figure}[h]
	\centering
	\includegraphics[width=0.5\textwidth]{files/2.png}
\end{figure}

Τα πακετα ακολουθουν τη διαδρομή (5\rightarrow 1 \rightarrow 2 \rightarrow 7)
για να φτιάσουν στον προορισμό τους. Η διαδρομη αυτη ειναι η συντομοτερη
δυνατη. Ακολουθείται και κατά τις δύο κατευθύνσεις και δεν υπάρχει συντομότερη
διαδρομή με την παρούσα συνδεσμολογία μιας και το κόστος στις ζεύξεις είναι
ίσο και σταθερό.


\inputminted[\size=\scriptsize]{tcl}{files/ex3_1.tcl}
\section
Σε αυτήν ενότητα θα δούμε τη διαφορά μεταξύ στατικής και δυναμικής
δρομολόγησης και την εκάστοτε αντιμετώπιση από το δίκτυο. Μεταβάλλουμε λίγο
τον κώδικα, διακόπτοντας την σύνδεση ανάμεσα σε 1 και 2 για 0.4sec.

\inputminted[\size=\scriptsize]{tcl}{files/ex3_2.tcl}

\begin{figure}[h]
	\centering
	\includegraphics[width=0.5\textwidth]{files/3.png}
    \caption{Για χρόνο 1.2sec η σύνδεση διακόπτεται στη ζεύξη 1-2.}
\end{figure}


Προκειμένου να διορθώσουμε το παραπάνω πρόβλημα, αλλάζουμε τη σύνδεση σε
δυναμική με την ακόλουθη εντολή:
\begin{minted}[\size=\scriptsize]{tcl}
%$ns rtproto DV
\end{minted}
Παρατηρούμε ορισμένες αλλαγές σε σχέση με πριν. Αρχικά, εξετάζεται με
δοκιμαστικά πακέτα η συνδεσιμότητα των γραμμών:
\begin{figure}[h]
	\centering
	\includegraphics[width=0.5\textwidth]{files/4.png}
\end{figure}

Μετά τη διακοπή της σύνδεσης 1-2 επιλέγεται εναλλακτική διαδρομή, όταν
παρατηρείται η πρώτη απώλεια πακέτων:

\begin{figure}[h]
	\centering
	\includegraphics[width=0.5\textwidth]{files/5.png}
\end{figure}

Τέλος, μόλις η σύνδεση ενεργοποιηθεί ξανά, αρχίζει και πάλι η μετάδοση πακέτων
από αυτή. Ασφαλώς υπάρχει και ένας χρόνος που μεσολαβεί κατά τον οποίο και οι
δύο διαδρομές έχουν πακέτα.

\begin{figure}[h]
	\centering
	\includegraphics[width=0.5\textwidth]{files/6.png}
\end{figure}

\begin{itermize}
\item Εξηγείστε γιατί με τη στατική δρομολόγηση, οι κόμβοι εξακολουθούν να
στέλνουν πακέτα μετά τη διακοπή της ζεύξης.

Καθώς, η δρομολόγηση είναι στατική, δεν ενημερώνονται οι κόμβοι με κανένα
τρόπο για τη διακοπή της σύνδεσης. Επομένως δεν υπάρχει κάποια ανατροφοδότηση
που θα τους οδηγούσε σε σχεδιασμό νέας διαδρομής.

\item Τα πακέτα που χάνονται θα ξαναμεταδοθούν από τους αντίστοιχους κόμβους,
όταν επανέλθει η σύνδεση;

Αφού δεν υπάρχει ενημέρωση σχετικά με την απώλεια τους, ασφαλώς και οι κόμβοι
δεν έχουν καμία πληροφορία για το αν χάθηκαν πακέτα και ποια είναι αυτά. Οπότε
τα συγκεκριμένα πακέτα δε θα μεταδοθούν ξανά όταν επανέλθει η σύνδεση.

\item Τι παρατηρείτε όταν γίνεται διακοπή ζεύξης και έχουμε δυναμική
δρομολόγηση; 

Περιγράψτε με απλά λόγια τη διαδικασία που λαμβάνει χώρα στο animation.
Στη δυναμική δρομολόγηση, υπάρχει διαρκής ενημέρωση σχετικά με τη
συνδεσιμότητα της κάθε γραμμής με τη βοήθεια βοηθητικών πακέτων. Αρχικά
επιλέγεται η διαδρομή 5-1-2-7 που είναι και η συντομότερη. Μόλις κοπεί η
σύνδεση 1-2, αμέσως βρίσκεται η συντομότερη διαδρομή που δε χρησιμοποιεί την
αποσυνδεμένη ζεύξη. Αυτή είναι η 5-1-0-3-7. Προφανώς η διαδρομή αυτή είναι πιο
αργή από την πρώτη. Για αυτό το λόγο, μόλις ενεργοποιηθεί ξανά η γραμμή 1-2,
αυτόματα η μεταφορά πακέτων αρχίζει να ξαναγίνεται από εκεί.

\item Ποιος από όλους τους κόμβους καθορίζει από ποια διαδρομή θα προωθηθούν
κάθε φορά τα πακέτα;

Αρχικά η πληροφορία για τη συνδεσιμότητα φθάνει στους κόμβους που έχουν επαφή
με την αποσυνδεμένη ζεύξη και εκεί επανακαθορίζεται η διαδρομή που θα
ακολουθηθεί για τον τελικό προορισμό. Σταδιακά η πληροφορία αυτή μεταβιβάζεται
και στους γειτονικούς κόμβους και μόλις φτάσει στον αρχικό κόμβο, αυτός
υπολογίζει ξανά τη διαδρομή λαμβάνοντας υπ’ όψιν τις αποσυνδεμένες ζεύξεις.
\item Για ποιο λόγο τα πακέτα ακολουθούν τις συγκεκριμένες διαδρομές αφότου
πέσει η σύνδεση;

Ο λόγος που γίνεται αυτό είναι επειδή οι συγκεκριμένες διαδρομές είναι οι
συντομότερες για τον προορισμό που δεν χρησιμοποιούν την κομμένη ζεύξη. Επί
της ουσίας, επανασχεδιάζεται η διαδρομή στον κόμβο που εντοπίζεται η διακοπή
ζεύξης και, όταν η πληροφορία φτάσει στον αρχικό κόμβο σε αυτόν, για να βρεθεί
ο συντομότερος τρόπος ώστε να προσεγγιστεί ο προορισμός χωρίς αυτή τη σύνδεση.
\item Θα μπορούσαν να δρομολογηθούν από άλλους κόμβους;

Ασφαλώς η δρομολόγηση θα μπορούσε να γίνει και από άλλους κόμβους, με
μεγαλύτερο πλήθος κόμβων να διαρρέεται αλλά κάτι τέτοιο είναι χρονοβόρο χωρίς
να προσφέρει επί της ουσίας κάτι.
\end{itemize}


\section
Σε αυτό το κομμάτι της άσκησης αλλάζουμε το κόστος των διαδρομών όπως φαίνεται
στον κώδικα παρακάτω.

\inputminted[\size=\scriptsize]{tcl}{files/ex3_3.tcl}

Εκτελώντας την προσομοίωση και πάλι παρατηρούμε ότι πλέον όταν κόβεται η ζεύξη
1-2 επιλέγεται διαφορετική πορεία από πριν. Καθώς  αυτή τη φορά
συνοπολογίζουμε και το διαφορετικό κόστος κάθε ζεύξης 5-0-3-7.

\begin{figure}[h]
	\centering
	\includegraphics[width=0.5\textwidth]{files/8.png}
\end{figure}






\begin{itemize}
\item Ποιές διαδρομές ακολουθούν τα πακέτα πριν , μετά και κατά τη διάρκεια της
πτώσης της σύνδεσης;

Τα πακέτα όπως ήταν αναμενόμενο ακολουθούν πριν και μετά την πτώση της
σύνδεσης 1-2 την ίδια διαδρομή με τα προηγούμενα ερωτήματα. Κατά τη διάρκεια
της πτώσης επιλέγουν η διαδρομή 5-1-0-4-3-7 .

\item Για ποιόν λόγο τα πακέτα ακολουθούν τις συγκεκριμένες διαδρομές;

Ο λόγος για τον οποίο τα πακέτα ακολουθούν τις παραπάνω διαδρομές είναι ότι με
βάση τα βάρη που έχουν οι ζεύξεις, αυτές είναι οι διαδρομές που κοστίζουν
λιγότερο.

\item Θα μπορούσαν να δρομολογηθούν από άλλους κόμβους;

Προφανώς θα μπορούσαν να δρομολογηθούν από άλλους κόμβους τα πακέτα (βλ. Μέσω
του κόμβου 8) αλλά αυτό θα ήταν ακριβότερο γι’αυτό επιλέγεται η παραπάνω
διαδρομή.
\end{itemize}






\end{document}

