%{{{ preamble
\documentclass[a4paper,9pt]{article}
\usepackage{anysize}
\marginsize{2cm}{2cm}{1cm}{1cm}
%\textwidth 6.0in \textheight = 664pt
\usepackage{xltxtra}
\usepackage{xunicode}
\usepackage{graphicx}
\usepackage{color}
\usepackage{xgreek}
\usepackage{fancyvrb}
\usepackage{minted}
\usepackage{listings}
\usepackage{enumitem} 
\usepackage{framed} 
\usepackage{relsize}
\usepackage{float} 
\usepackage{pstricks}
\usepackage{pst-node}
\usepackage{pst-blur}
\setmainfont[Mapping=tex-text]{FreeSerif}
%}}}
\begin{document}

\def\thesection {\Roman{section}.}

\include{title/title}

\section*{Aνάλθση αποτελεσμάτων με τη βοήθεια του NAM}
Αρχικά χρησιμοποιήσαμε τον κώδικα που μας δίνεται από την εκφώνηση:

\inputminted[fontsize=\footnotesize]{tcl}{files/ex4_1.tcl}

Τρέχουμε το animation και παρατηρούμε την εξέλιξη της πορείας μετάδοσης των
πακέτων:
\begin{itemize}
    \item Με τη βοήθεια του animation παρατηρούμε πως η αποστολή των 50
    πακέτων από τον κόμβο 0 στον κόμβο 3 (Go back N protocol window 7) ολοκληρώνεται τη
    στιγμή 1.3sec. Από την άλλη πλευρά το τελευταίο πακέτο από τον κόμβο 1
    στον κόμβο 2 (Stop and Wait protocol window 1) φτάνει τη χρονική στιγμη
    5.6sec.
    \item Κάθε στιγμή μπορούν να αποσταλούν χωρίς απώλειες
    $\frac{2Mb}{1040*8bits}\approx240packets$.  Συμπεραίνουμε λοιπόν ότι και
    τα 50 πακέτα μπορούν να σταλούν σε ένα μόνο παράθυρο. ΄Ετσι το ελάχιστο μέγεθος παραθύρου που εξασφαλίζει τον
   μικρότερο χρόνο μετάδοσης είναι 50.
   \begin{minted}{tcl}
   $tcp0 set window_ 50
   $tcp0 set windowInit_ 50
   \end{minted}
   Αυτή τη φορά ο χρόνος μετάδοσης είναι 0.26sec (oloasdf

\end{itemize}

\end{document}

