%{{{ preamble
\documentclass[a4paper,9pt]{article}
\usepackage{anysize}
\marginsize{2cm}{2cm}{1cm}{1cm}
%\textwidth 6.0in \textheight = 664pt
\usepackage{xltxtra}
\usepackage{xunicode}
\usepackage{graphicx}
\usepackage{color}
\usepackage{xgreek}
\usepackage{fancyvrb}
\usepackage{minted}
\usepackage{listings}
\usepackage{enumitem} 
\usepackage{framed} 
\usepackage{relsize}
\usepackage{float} 
\usepackage{pstricks}
\usepackage{pst-node}
\usepackage{pst-blur}
\setmainfont[Mapping=tex-text]{FreeSerif}
%}}}
\begin{document}

\def\thesection {\Roman{section}.}

\include{title/title}

\section*{Aνάλθση αποτελεσμάτων με τη βοήθεια του NAM}
Αρχικά χρησιμοποιήσαμε τον κώδικα που μας δίνεται από την εκφώνηση:

\inputminted[fontsize=\footnotesize]{tcl}{files/ex4_1.tcl}

Τρέχουμε το animation και παρατηρούμε την εξέλιξη της πορείας μετάδοσης των
πακέτων:
\begin{itemize}
    \item Με τη βοήθεια του animation παρατηρούμε πως η αποστολή των 50
        πακέτων από τον κόμβο 0 στον κόμβο 3 (Go back N protocol window 7) ολοκληρώνεται τη
        στιγμή 1.3sec. Από την άλλη πλευρά το τελευταίο πακέτο από τον κόμβο 1
        στον κόμβο 2 (Stop and Wait protocol window 1) φτάνει τη χρονική στιγμη
        5.6sec.
    \item Κάθε στιγμή μπορούν να αποσταλούν χωρίς απώλειες
        $\frac{2Mb}{1040*8bits}\approx240packets$.  Συμπεραίνουμε λοιπόν ότι και
        τα 50 πακέτα μπορούν να σταλούν σε ένα μόνο παράθυρο. ΄Ετσι το ελάχιστο μέγεθος παραθύρου που εξασφαλίζει τον
        μικρότερο χρόνο μετάδοσης είναι 50.
        \begin{minted}{tcl}
            $tcp0 set window_ 50
            $tcp0 set windowInit_ 50
        \end{minted}
        Αυτή τη φορά ο χρόνος μετάδοσης είναι 0.26sec (0.76-0.5)
    \item Εκτελώντας ακόμη μία φορά την προσομοίωση με καθυστέρηση 500ms στη
        ζεύξη 0-3 παρατηρούμε πως η μετάδοση ολοκληρώνεται τη στιγμή 1.2sec.
\end{itemize}

\section*{Ανάλυση αποτεέσμάτων με τη βοήθεια αρχείου ίχνους}
\inputminted[fontsize=\footnotesize]{awk}{files/script_1.awk}

\noindent Output:
\inputminted[fontsize=\footnotesize]{text}{files/awk.out}

\begin{itemize}
    \item Από την έξοδο του script παρατηρούμε πως για κάθε ροή κίνησης
        παραλαμβάνονται 50 πακέτα $=$ 51960bytes. 
    \item Από το αρχείο ίχνους παρατηρούμε πως η μετάδοση ανάμεσα στους
        κόμβους 0,3 σταματάει τη στιγμή 1.3sec ενώ μεταξύ των κόμβων 1,2 τη
        στιγμή 5.6sec.
\end{itemize}

\begin{enumerate}
    \item 
        \begin{itemize}
            \item Ρυθμός μετάδοσης: $\frac{51960*8}{1.3-0.5}=519600bps$
            \item Χρησιμοποίηση καναλιού: $\frac{519600}{2*2^{20}} = 0.247765
                \approx 24.8\% $
        \end{itemize}
    \item 
        \begin{itemize}
            \item Ρυθμός μετάδοσης: $\frac{51960*8}{5.6-0.5}=81505.9bps$
            \item Χρησιμοποίηση καναλιού: $\frac{81505.9}{2*2^{20}} = 0.038865
                \approx 3.9\% $
        \end{itemize}
\end{enumerate}
\end{document}

