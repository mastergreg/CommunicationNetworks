%{{{ preamble
\documentclass[a4paper,9pt]{article}
\usepackage{anysize}
\marginsize{2cm}{2cm}{1cm}{1cm}
%\textwidth 6.0in \textheight = 664pt
\usepackage{xltxtra}
\usepackage{xunicode}
\usepackage{graphicx}
\usepackage{color}
\usepackage{xgreek}
\usepackage{fancyvrb}
\usepackage{minted}
\usepackage{listings}
\usepackage{enumitem} 
\usepackage{framed} 
\usepackage{relsize}
\usepackage{float} 
\usepackage{pstricks}
\usepackage{pst-node}
\usepackage{pst-blur}
\setmainfont[Mapping=tex-text]{FreeSerif}
%}}}
\begin{document}

\def\thesection {\Roman{section}.}

\include{title/title}
\section{Ερωτήσεις}
\begin{itemize}
    \item Ο ρυθμός αποστολής είναι $1.2\times10^6 bps$.
    \item Συνολικά μεταφέρθηκαν $1.2\times10^6 bytes = 9.6*10^6 bits$ από την αρχή
        ως το τέλος της προσομοίωσης.
    \item Στη γραμμή κάθε στιγμή βρίσκονται 1500 bytes όσο και το μέγεθος του
        κάθε πακέτου.
        \begin{figure}[h]
            \centering
            \includegraphics[width=0.5\textwidth]{files/1packet.png}
        \end{figure}
    \item Στη γραμμή κάθε στιγμή βρίσκονται $2\times1500$ bytes όσο και το μέγεθος του
        2 πακέτων.
        \begin{figure}[h]
            \centering
            \includegraphics[width=0.5\textwidth]{files/2packet.png}
        \end{figure}
    \item Κάθε πακέτο φέρει δεδομένα μεγέθους 1460 bytes συνεπώς
        $1460\times100 = 146000$ bytes/sec $\times 8 = 1168000$ bps
    \item Για να πετύχουμε ροή δεδομένων
        $1.2Mbps=1.2\times2^{20}bps=1.2\times2^{17}bytes$ πρέπει να στέλνουμε
        $1.2\times2^{17}\div1460$ πακέτα/sec συνεπώς πρέπει ο ρυθμός αποστολής
        να αυξηθεί, δηλαδή να μειωθεί το interval πράγμα που το πετυχαίνουμε
        με την ακόλουθη εντολή:
\begin{minted}{tcl}
$cbr0 set interval_ 0.0092
\end{minted}

\end{itemize}

\end{document}

