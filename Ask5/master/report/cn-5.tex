%{{{ preamble
\documentclass[a4paper,9pt]{article}
\usepackage{anysize}
\marginsize{2cm}{2cm}{1cm}{1cm}
%\textwidth 6.0in \textheight = 664pt
\usepackage{xltxtra}
\usepackage{xunicode}
\usepackage{graphicx}
\usepackage{color}
\usepackage{xgreek}
\usepackage{fancyvrb}
\usepackage{minted}
\usepackage{listings}
\usepackage{enumitem} 
\usepackage{framed} 
\usepackage{relsize}
\usepackage{float} 
\usepackage{pstricks}
\usepackage{pst-node}
\usepackage{pst-blur}
\setmainfont[Mapping=tex-text]{FreeSerif}
%}}}
\begin{document}

\def\thesection {\Roman{section}.}
\def\thesubsection {\Roman{subsection}.}

\include{title/title}

\section*{Yπολογισμός των πακέτων και της ποσότητας των δεδομένων που
ελήφθησαν}

Χρησιμοποιήσαμε τον κώδικα που μας δίνεται από την εκφώνηση:

\inputminted[fontsize=\footnotesize,linenos,frame=leftline]{tcl}{files/ex5_1.tcl}

Καθώς και το awk script:

\inputminted[fontsize=\footnotesize,linenos,frame=leftline]{awk}{files/script_1.awk}

Και πήραμε την ακόλουθη έξοδο:

\inputminted[fontsize=\footnotesize,linenos,frame=leftline]{text}{files/awk.out}

\section*{Μελέτη της απόδοσης της Selective Repeat}

\subsection{}
Για το θεωρητικό υπολογισμό χρησιμοποιούμε την εξίσωση:
\[\eta=min\left\{ \frac{W \times TRANSP}{S},1\right\} \]
\begin{itemize}
    \item $S = TRASP + 2 \times PROP + TRANSA$
    \item $TRANSP$ χρόνος μετάδοσης του πακέτου, (μήκος)/(ρυθμός μετάδοσης)
    \item $TRANSA$ χρόνος μετάδοσης της επαλήθευσης
    \item $PROP$ καθυστέρηση διάδοσης του πακέτου
    \item $W$ μήκος παραθύρου
\end{itemize}


\begin{itemize}
    \item $TRANSP = \frac{1500 \times 8}{10^7} = 1.2 \times ms$
    \item $TRANSA = \frac{40 \times 8}{10^7} = 32 \times \mu s$
    \item $W=4$
    \item $PROP = 5 ms$
    \item $S = (1.2 + 10 + 0.032) \times 10^{-3} = 11.232 ms$
\end{itemize}

Συνεπώς η θεωρητική τιμή της απόδοσης υπολογίζεται:
\[\eta=min\left\{ \frac{4 \times 1.2}{11.232},1\right\} = 0.4274\]

Από την εκτέλεση του awk script υπολογίζουμε καθαρό ρυθμό μετάδοσης $\frac{1644680
\times 8}{3} = 4.386Mbps $

\[\eta=\frac{4.386 \times 10^6}{10^7} = 0.4386 \]

Η πραγματική απόδοση είναι πολύ κοντά στην θεωρητική απόδοση.

\subsection{}
Κατά την προσομοίωση στάλθηκαν 1644680 bytes σε 1068 πακέτα. Αυτά
αποστέλλονται σε χρόνο 3s έτσι υπολογίζουμε το ρυθμό μετάδοσης:
\[\frac{1644680 \times 8}{3} = 4.386Mbps \]

Και χρησιμοποιήση καναλιού:
\[\eta=\frac{4.386 \times 10^6}{10^7} = 0.4386 \]

\subsection{}
Συνυπολογίζοντας στο μέγεθος των πακέτων το μήκος των επικεφαλίδων TCP και IP
έχουμε καλύτερη προσέγγιση καθώς αυτές προσμετρώνται στο συνολικό μέγεθος που
θα μεταδοθούν πάνω στη ζεύξη.

\begin{itemize}
    \item $TRANSP = \frac{1540 \times 8}{10^7} = 1.232 \times ms$
    \item $TRANSA = \frac{40 \times 8}{10^7} = 32 \times \mu s$
    \item $W=4$
    \item $PROP = 5 ms$
    \item $S = (1.232 + 10 + 0.032) \times 10^{-3} = 11.264 ms$
\end{itemize}

\[\eta=min\left\{ \frac{4 \times 1.232}{11.264},1\right\} = 0.4375\]

\subsection{}
Για να μεγιστοποιηθεί η απόδοση θα πρέπει $\eta=1$. 

\begin{itemize}
    \item $TRANSP = \frac{L \times 8}{10^7} = L \times 8 \times 10^{-7} s$
    \item $S = L\times8\times10^{-7}+ 10^{-2} + 0.032 \times 10^{-3} =
        L\times8\times10^{-7}+10.032\times10^{-3} s$
\end{itemize}


\[\eta=\frac{4 \times
L\times8\times10^{-7}}{L\times8\times10^{-7}+10.032\times10^{-3} } = 1\]

\[ \Rightarrow L = \frac{10.32\times10^4}{3\times8} = 4300bytes\]

Συνεπώς καθαρό μέγεθος πακέτου χωρίς τις επικεφαλίδες είναι
$4300-40=4260bytes$.

Τρέχοντας πάλι την προσομοίωση με το νέο μέγεθος πακέτου έχουμε έξοδο:

\inputminted[fontsize=\footnotesize,linenos,frame=leftline]{text}{files/awk_2.out}

\[\eta=\frac{\frac{3753860 \times 8}{3}}{10^7} \approx 1 \]

\subsection{}
Αυτή τη φορά θα έχουμε σταθερό μέγεθος πακέτου αλλά δεκαπλάσιο ρυθμό
μετάδοσης.

\begin{itemize}
    \item $TRANSP = \frac{1540 \times 8}{10^8} = 1.232 10^{-4} s$
    \item $TRANSA = \frac{40 \times 8}{10^7} = 3.2 \mu s$
    \item $PROP = 5 ms$
    \item $S = (1.232 + 100 + 0.032) \times 10^{-4} = 101.264 ms$
\end{itemize}

\[\eta=\frac{W \times 1.232}{101.264} = 1 \Rightarrow W = 82\]

Από την έξοδο του awk script:
\inputminted[fontsize=\footnotesize,linenos,frame=leftline]{text}{files/awk_3.out}
\[\eta=\frac{\frac{32808120 \times 8}{3}}{100Mbps} \approx 0.874\]

Για μήκος παραθύρου 82 ο αριθμός ακολουθίας είναι 163. Κατά συνέπεια
χρειαζόμαστε 8 bits για την αναπαράσταση.

\subsection{}
Για μέγεθος παραθύρου 82, καθυστέρηση γραμμής 50ms, μήκος πακέτου 1500bytes,
και ρυθμό μετάδοσης 100Mbps απο την προσομοίωση έχουμε τα εξής:
\inputminted[fontsize=\footnotesize,linenos,frame=leftline]{text}{files/awk_4.out}

\[\eta=\frac{\frac{1764800 \times 8}{3}}{100Mbps} \approx 0.047\]

\subsection{}
Με χρήση του προτοκόλου Go Back N 
\begin{minted}{tcl}
set tcp0 [new Agent/TCP/Reno]
\end{minted}
παίρνουμε την παρακάτω έξοδο:
\inputminted[fontsize=\footnotesize,linenos,frame=leftline]{text}{files/awk_5.out}

\[\eta=\frac{\frac{1644680 \times 8}{3} \times 10^6}{10^7} = 0.4386 \]

Η απόδοση και των δύο προτοκόλλων είναι ίδια καθώς δεν έχουμε απώλειες πακέτων
κατά τη μεταφορά.

\end{document}

